\documentclass[10pt,fleqn]{article} % Default font size and left-justified equations
\usepackage[%
    pdftitle={Modélisation systèmes multiphysiques : Modélisation linéaire et non linéaire},
    pdfauthor={Xavier Pessoles}]{hyperref}
    
\input{style/new_style}
\input{style/macros_SII}
\usepackage{multicol}
\usepackage{siunitx}
\fichetrue
%\fichefalse

\proftrue
%\proffalse

\tdtrue
%\tdfalse

\courstrue
\coursfalse

\def\discipline{Sciences \\Industrielles de \\ l'Ingénieur}
\def\xxtete{Sciences Industrielles de l'Ingénieur}

\def\classe{PSI$\star$ -- MP}
\def\xxnumpartie{\textsf{Cycle 01}}
\def\xxpartie{Modéliser le comportement linéaire et non linéaire des systèmes multiphysiques}


\def\xxnumchapitre{Chapitre 1 \vspace{.2cm}}
\def\xxchapitre{\hspace{.12cm} Modélisation multiphysique}


\def\xxtitreexo{QCM}%Motorisation du moteur Haibike}
\def\xxsourceexo{\hspace{.2cm} \footnotesize{Pôle Chateaubriand Joliot-Curie.}}


\def\xxposongletx{2}
\def\xxposonglettext{1.45}
\def\xxposonglety{20}
%\def\xxonglet{Part. 1 -- Ch. 3}
\def\xxonglet{\textsf{Cycle 01}}

\def\xxactivite{QCM 01}
\def\xxauteur{\textsl{Pôle Chateaubriand Joliot-Curie.}}

\def\xxcompetences{%
\textsl{%
\textbf{Savoirs et compétences :}\\
%Les sources sont associées par un \emph{hacheur série}. La détermination des grandeurs électriques associées à ce montage permet de conclure vis à vis du cahier des charges.
%\noindent \textbf{Résoudre :} à partir des modèles retenus :
%\begin{itemize}[label=\ding{112},font=\color{ocre}] 
%\item choisir une méthode de résolution analytique, graphique, numérique;
%\item mettre en \oe{}uvre une méthode de résolution.
%\end{itemize}
%\begin{itemize}[label=\ding{112},font=\color{ocre}] 
%\item \textit{Rés -- C1.1 :} Loi entrée sortie géométrique et cinématique -- Fermeture géométrique.
%\end{itemize}
%
%\noindent \textit{Mod2 -- C4.1 :} Représentation par schéma bloc.
}}

\def\xxfigures{
%\includegraphics[width=.9\linewidth]{images/c-evolution}
}%figues de la page de garde


\def\xxpied{%
Cycle 01 -- Modéliser le comportement des systèmes multiphysiques\\
Chapitre 1 -- \xxactivite%
}

\setcounter{secnumdepth}{5}
%---------------------------------------------------------------------------

\usepackage{pgfplots}
\begin{document}
\def\pathfig{images}
%\chapterimage{png/Fond_Cin}
\input{style/new_pagegarde}
\vspace{4cm}
\pagestyle{fancy}
\thispagestyle{plain}

\def\columnseprulecolor{\color{ocre}}
\setlength{\columnseprule}{0.4pt} 

\def\pathfig{images}
On donne les modèles ci-dessous sur lesquels les sphères représentent des
masses ponctuelles situées en leurs centres et dont les masses sont
proportionnelles au rayon des sphères. On néglige la masse et l’inertie de l’axe
de rotation et des tiges qui relient les masses à l’axe.
Dans chacun des cas suivant, indiquer si le système est équlibré statiquement ou non, dynamiquement, ou non.

\begin{multicols}{3}


\begin{center}
\includegraphics[height=3.5cm]{images/fig_01}
\end{center}

\begin{center}
\includegraphics[height=3.5cm]{images/fig_02}
\end{center}


\begin{center}
\includegraphics[height=3.5cm]{images/fig_03}
\end{center}


\begin{center}
\includegraphics[height=3.5cm]{images/fig_04}
\end{center}


\begin{center}
\includegraphics[height=3.5cm]{images/fig_05}
\end{center}


\begin{center}
\includegraphics[height=3.5cm]{images/fig_06}
\end{center}


\begin{center}
\includegraphics[height=3.5cm]{images/fig_07}
\end{center}


\begin{center}
\includegraphics[height=3.5cm]{images/fig_08}
\end{center}


\begin{center}
\includegraphics[height=3.5cm]{images/fig_09}
\end{center}


\begin{center}
\includegraphics[height=3.5cm]{images/fig_10}
\end{center}


\begin{center}
\includegraphics[height=3.5cm]{images/fig_11}
\end{center}


\begin{center}
\includegraphics[height=3.5cm]{images/fig_12}
\end{center}



\end{multicols}







\end{document}